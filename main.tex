%%%%%%%%%%%%%%%%%%%%%%%%%%%%%%%%%%%%%%%%%
% Plasmati Graduate CV 
% LaTeX Template
% Version 1.0 (24/3/13)
%
% This template has been downloaded from:
% http://www.LaTeXTemplates.com
%
% Original author:
% Alessandro Plasmati (alessandro.plasmati@gmail.com)
%
% License:
% CC BY-NC-SA 3.0 (http://creativecommons.org/licenses/by-nc-sa/3.0/)
%
% Important note:
% This template needs to be compiled with XeLaTeX.
% The main document font is called Fontin and can be downloaded for free
% from here: http://www.exljbris.com/fontin.html
%
%%%%%%%%%%%%%%%%%%%%%%%%%%%%%%%%%%%%%%%%%

%----------------------------------------------------------------------------------------
%	PACKAGES AND OTHER DOCUMENT CONFIGURATIONS
%----------------------------------------------------------------------------------------

% \documentclass[a4paper,10pt]{extarticle} % Default font size and paper size

% \usepackage{fontspec} % For loading fonts
% \defaultfontfeatures{Mapping=tex-text}
% \setmainfont[Path = ./Ubuntu/,  %% Optional; but UPDATE this if 
%  Extension = .ttf,
%  UprightFont = *-Regular,
%  BoldFont = *-Bold,
%  ItalicFont = *-Italic,
%  SmallCapsFont = *-Medium]
% {Ubuntu}
% \fontspec{[fontawesome-webfont.ttf]}
% \usepackage{color}
% \definecolor{primary}{RGB}{91,12,43}
% \definecolor{secondary}{RGB}{55, 1, 42}
% \definecolor{extra}{RGB}{6, 10, 74}
% \usepackage{xunicode,xltxtra,url,parskip} % Formatting packages

% \usepackage[usenames,dvipsnames]{xcolor} % Required for specifying custom colors

% \usepackage{geometry}
% \geometry{a4paper,margin=0.40cm}

% \usepackage{hyperref} % Required for adding links	and customizing them
% \definecolor{linkcolour}{rgb}{0.25,0.30,0.34} % Link color
% \hypersetup{colorlinks,breaklinks=linkcolour,urlcolor=linkcolour,linkcolor=linkcolour} % Set link colors throughout the document

% \usepackage{titlesec} % Used to customize the \section command
% \titleformat{\section}{\large\scshape\raggedright}{}{0em}{}[\titlerule] % Text formatting of sections
% \titlespacing{\section}{0pt}{0pt}{0pt} % Spacing around sections

% \usepackage{multicol}
% \setlength{\columnsep}{0cm}

% \usepackage{tabularx}

% \usepackage{textcomp}

% \usepackage{fontawesome}

% \usepackage{enumitem}
% \setlist[description]{
%   topsep=10pt,             % space before start / after end of list
%   itemsep=1pt,          % space between items
% }

% \def\arraystretch{1}
% \renewcommand{\baselinestretch}{1.1}

% \begin{document}

% \pagestyle{empty} % Removes page numbering

% %----------------------------------------------------------------------------------------
% %	NAME AND CONTACT INFORMATION
% %----------------------------------------------------------------------------------------
% \begin{multicols}{3}
% % \normalsize  \faGlobe\ {\href{http://ghostwriternr.me/}{\  ghostwriternr.me}}\\
% \normalsize \faGithub\ {\href{https://github.com/TheLethalCode}{\  TheLethalCode}}\\
% \normalsize \faHome {\href{https://codeforces.com/profile/TheLethalCode}{\ \space TheLethalCode (Codeforces)}}\\
% % \normalsize \faHome {\href{https://www.codechef.com/users/thelethalcode}{\ \space TheLethalCode (CodeChef)}}\\
% \columnbreak
% \normalsize\par{\centering{\huge\textsc{\textcolor{primary}{Kousshik Raj} M}}\par} % Your name
% \par{\centering\normalsize {\textsc{A-107, JCB Hall of Residence, }}\hfill\par}
% \vspace{-0.2cm}
% \par{\centering\normalsize {\textsc{IIT Kharagpur, WB, India - 721302 }}\hfill\par}

% \columnbreak
% \raggedright\hfill\normalsize  \faLinkedinSquare\ {\href{https://www.linkedin.com/in/kousshik-raj-murali-b3976916a/}{\  Kousshik Raj}}\\
% \raggedright\hfill\normalsize \faEnvelope\ {\href{mailto:kousshikraj.raj@gmail.com}{\  kousshikraj.raj@gmail.com}}\\
% \raggedright\hfill{\faPhone\ \  +91-9626888198}
% \end{multicols}

% %----------------------------------------------------------------------------------------
% %	EDUCATION
% %----------------------------------------------------------------------------------------

% \vspace{-0.7cm}
% \section{\textcolor{primary}{Education}}
% \begin{tabular}{r|p{17.5cm}}	
% 2017-2022 & B.Tech + M.Tech (Dual degree) in \textbf{Computer Science and Engineering}\hfill\textsc{GPA: } 9.33/10.0 (* Ongoing)\\
% \textsc{(Expected)}& Indian Institute of Technology, Kharagpur \hfill\\
% 2015-2017 & Higher Secondary School Certificate Examination, \textbf{CBSE}\hfill\textsc{Percentage: } 96\% \\
% & Maharishi International Residential School, Kancheepuram \hfill\\
% % 2015 & Secondary School Certificate Examination, \textbf{CBSE}\hfill\textsc{CGPA: } 10/10 \\
% % & Mahatma Gandhi Centenary Vidyalaya, Trichy \hfill\\
% \end{tabular}

% %----------------------------------------------------------------------------------------
% %	SKILLS 
% %----------------------------------------------------------------------------------------

% \vspace{-0.1cm}
% \section{\textcolor{primary}{Technical Skills}}
% \begin{tabular}{r|p{15cm}}
% \textsc{Programming Languages} & C,\ C++,\ Python,\ GoLang,\ JavaScript,\ Julia \\
% \textsc{Libraries / Frameworks} & ROS, \ Selenium,\ STL,\ OpenCV,\ Numpy,\ Requests,\ Flask \\
% \textsc{Databases} & MySQL, SQLite\\
% \textsc{Systems / Platforms} & Linux,\ Docker,\ Android,\ Windows,\ Git \\
% % \textsc{Markup / Templating} & HTML, CSS \\
% \textsc{Others} & Bash,\ Latex,\ Solidworks \\
% \end{tabular}

% %----------------------------------------------------------------------------------------
% %	Experience
% %----------------------------------------------------------------------------------------
% \vspace{-0.15cm}
% \section{\textcolor{primary}{Research Experience}}
% \begin{tabularx}{\linewidth}{ l | X }
% \textsc{Feb 18} & \textbf{Artificial Intelligence Team Member} \hfill\href{http://www.agv.iitkgp.ac.in/}{\textbf{Autonomous Ground Vehicle Research Group}}\\
% \textsc{Present} & {- Working as a software stack team member, tackling the various challenges faced to model a complete autonomous vehicle capable of traversing dynamic environments.}\\
% & {- Working on the various aspects of path planning from a source to one or more destinations through an ever-changing surrounding  and their run-time optimization.}\\
% & {- Working on a high accuracy lane detection module for an outdoor environment with unfavourable conditions.}
% \end{tabularx}

% %----------------------------------------------------------------------------------------
% %	Projects
% %----------------------------------------------------------------------------------------

% \vspace{-0.1cm}
% \section{\textcolor{primary}{Projects}}
% \vspace{-0.6cm}
% \begin{tabular}{p{19.7cm}}
% \begin{description}[style=nextline, font=$\bullet$\hspace{2mm}\normalsize]
%  \item[\href{https://github.com/TheLethalCode/opensoft18}{DigiCon},\space OpenSoft 2018 IIT Kharagpur]
%   This web application accurately parses and mines the contents of a hand-written doctor's prescription and segregates the medicines along with their doses while checking for possible errors and lists them out in a more readable fashion.
%   \item[\textcolor{extra}{Hybrid A-Star \& DWA},\space Path Planning Algorithms] Improved, parallelized and novel implementation of the conventional Hybrid A-star global path planner allowing for kinetic constraints of the bot, capable of running at 7Hz in a moderately populated environment. An enhanced objective function realized for the Dynamic Window Approach local path planner which results in a shorter path traversal time.
%  \item[\textcolor{extra}{Eklavya 6.0}, \space Intelligent Ground Vehicle Competition (IGVC) 2018]
%  A robot capable of intelligently traversing an obstacle ridden course with the help of visual and sensory input. The bot took part in the \href{http://www.igvc.org/}{\textbf{IGVC 2018}} and bagged the 2nd place.
%  \item[\href{https://github.com/TheLethalCode/Artemis-arrow}{Artemis' Arrow},\space A Web Application] A web app that tries to retrieve various forms of entertainment such as songs, books, anime from throughout the web and offers it at a single place while offering multiple user customizations and features. 
%  \item[\href{https://github.com/TheLethalCode/brkout}{BrkOut},\space A game made using PyGame]
%  An interactive game that incorporates real time collision and momentum conservation in a graphical interface made using Pygame in Python. It also uses a basic encryption which emphasizes the prison-breaking theme of the game.
% \end{description}
% \end{tabular}

% %----------------------------------------------------------------------------------------
% %	COURSEWORK
% %----------------------------------------------------------------------------------------

% \vspace{-0.6cm}
% \section{\textcolor{primary}{Related Courses} 
% \hfill{\normalsize{* Currently Studying}}}
% \vspace{-2.5mm}
% \begin{itemize}
% \begin{multicols}{3}
% \setlength{\itemsep}{-5pt}
%  \item Programming and Data Structures
%  \item Algorithms and Data Structures
%  \item Discrete Structures
%  \item Software Engineering*
%  \item Formal Language\&Automata Theory*
%  \item Probability and Statistics*
%  \item Switching Circuits*
%  \item Computer Vision*
%  \item Introduction to Machine Learning*
%  \end{multicols}
% \end{itemize}

% %----------------------------------------------------------------------------------------
% %	INTERESTS   
% %----------------------------------------------------------------------------------------
% \vspace{-0.5cm}
% \section{\textcolor{primary}{Interests}}
% \begin{description}
% Algorithms and Data Structures, Machine Learning, Computer Vision, Number Theory, Cryptography and Networking.
% \end{description}

% %----------------------------------------------------------------------------------------
% %	ACHIEVEMENTS AND INVOLVEMENTS
% %----------------------------------------------------------------------------------------
% \vspace{-0.4cm}
% \section{\textcolor{primary}{Achievements \& Involvements}}
% \vspace{-0.6cm}
% \begin{tabular}{p{19.7cm}}
% \begin{description}[font=$\bullet$\hspace{2mm}\normalsize]
%  \item[\href{https://kwoc.kossiitkgp.org/}{Kharagpur Winter of Code -}]One of the organizers of the five week long GSOC-styled Open Source Program and an active mentor of one of the projects in it. Responsible for the development and maintenance of the KWOC website.  
%  \item[\textcolor{extra}{Programming Societies -}] Co-founder of \href{https://www.facebook.com/codestashkgp/}{\textbf{CodeStash, IIT KGP}} and Core-Team Member of societies such as \href{https://kossiitkgp.in/}{\textbf{Kharagpur Open Source Societies }}and \href{https://www.facebook.com/CodeClub.IITKGP/}{\textbf{CodeClub}} where we help to capture, nurture and preserve the programming zeal that bubbles among the budding KGP students by organizing workshops, hackathons, fests, etc.
%  \item[\textcolor{extra}{Scholastic Achievements}]\textbf{}
%  \newline - \space \textbf{AIR 322 - JEE Advanced} (99.8 percentile) \newline - \space \textbf{AIR 1784 - JEE Mains} (99.8 percentile) \newline - \space Twice \href{http://www.kvpy.iisc.ernet.in/main/index.htm}{\textbf{Kishore Vaigyanic Protsahan Yojana (KVPY)}} Scholar
% \end{description}
% \end{tabular}
% \end{document}
\documentclass[a4paper,10pt]{extarticle} % Default font size and paper size

\usepackage{fontspec} % For loading fonts
\defaultfontfeatures{Mapping=tex-text}
\setmainfont[Path = ./Ubuntu/,  %% Optional; but UPDATE this if 
 Extension = .ttf,
 UprightFont = *-Regular,
 BoldFont = *-Bold,
 ItalicFont = *-Italic,
 SmallCapsFont = *-Medium]
{Ubuntu}
\fontspec{[fontawesome-webfont.ttf]}
\usepackage{color}
\definecolor{primary}{RGB}{91,12,43}
\definecolor{secondary}{RGB}{55, 1, 42}
\definecolor{extra}{RGB}{6, 10, 74}
\usepackage{xunicode,xltxtra,url,parskip} % Formatting packages

\usepackage[usenames,dvipsnames]{xcolor} % Required for specifying custom colors

\usepackage{geometry}
\geometry{a4paper,margin=0.40cm}

\usepackage{hyperref} % Required for adding links	and customizing them
\definecolor{linkcolour}{rgb}{0.25,0.30,0.34} % Link color
\hypersetup{colorlinks,breaklinks=linkcolour,urlcolor=linkcolour,linkcolor=linkcolour} % Set link colors throughout the document

\usepackage{titlesec} % Used to customize the \section command
\titleformat{\section}{\large\scshape\raggedright}{}{0em}{}[\titlerule] % Text formatting of sections
\titlespacing{\section}{0pt}{0pt}{0pt} % Spacing around sections

\usepackage{multicol}
\setlength{\columnsep}{0cm}

\usepackage{tabularx}

\usepackage{textcomp}

\usepackage{fontawesome}

\usepackage{enumitem}
\setlist[description]{
  topsep=10pt,             % space before start / after end of list
  itemsep=1pt,          % space between items
}

\def\arraystretch{1}
\renewcommand{\baselinestretch}{1.1}

\begin{document}

\pagestyle{empty} % Removes page numbering

%----------------------------------------------------------------------------------------
%	NAME AND CONTACT INFORMATION
%----------------------------------------------------------------------------------------
\begin{multicols}{3}
% \normalsize  \faGlobe\ {\href{http://ghostwriternr.me/}{\  ghostwriternr.me}}\\
\normalsize \faGithub\ {\href{https://github.com/TheLethalCode}{\  TheLethalCode}}\\
\normalsize \faHome {\href{https://codeforces.com/profile/TheLethalCode}{\ \space TheLethalCode (Codeforces)}}\\
% \normalsize \faHome {\href{https://www.codechef.com/users/thelethalcode}{\ \space TheLethalCode (CodeChef)}}\\
\columnbreak
\normalsize\par{\centering{\huge\textsc{\textcolor{primary}{Kousshik Raj} M}}\par} % Your name
\par{\centering\normalsize {\textsc{A-107, JCB Hall of Residence, }}\hfill\par}
\vspace{-0.2cm}
\par{\centering\normalsize {\textsc{IIT Kharagpur, WB, India - 721302 }}\hfill\par}

\columnbreak
\raggedright\hfill\normalsize  \faLinkedinSquare\ {\href{https://www.linkedin.com/in/kousshik-raj-murali-b3976916a/}{\  Kousshik Raj}}\\
\raggedright\hfill\normalsize \faEnvelope\ {\href{mailto:kousshikraj.raj@gmail.com}{\  kousshikraj.raj@gmail.com}}\\
\raggedright\hfill{\faPhone\ \  +91-9626888198}
\end{multicols}

%----------------------------------------------------------------------------------------
%	EDUCATION
%----------------------------------------------------------------------------------------

\vspace{-0.1cm}
\section{\textcolor{primary}{Education}}
\begin{tabular}{r|p{17.5cm}}	
2017-2022 & B.Tech + M.Tech (Dual degree) in \textbf{Computer Science and Engineering}\hfill\textsc{GPA: } 9.33/10.0 (* Ongoing)\\
\vspace{0.2cm}

\textsc{(Expected)}& Indian Institute of Technology, Kharagpur \hfill\\
2015-2017 & Higher Secondary School Certificate Examination, \textbf{CBSE}\hfill\textsc{Percentage: } 96\% \\
\vspace{0.2cm}

& Maharishi International Residential School, Kancheepuram \hfill\\
2015 & Secondary School Certificate Examination, \textbf{CBSE}\hfill\textsc{CGPA: } 10/10 \\
& Mahatma Gandhi Centenary Vidyalaya, Trichy \hfill\\
\end{tabular}

%----------------------------------------------------------------------------------------
%	SKILLS 
%----------------------------------------------------------------------------------------

\vspace{0.3cm}
\section{\textcolor{primary}{Technical Skills}}
\begin{tabular}{r|p{15cm}}
\textsc{Programming Languages} & C,\ \ C++,\ \ Python,\ \ Java,\ \ GoLang,\ \  Verilog\\
\textsc{Libraries / Frameworks} & ROS,\ \ OpenCV,\ \ Tenserflow,\ \ STL,\ \ Numpy,\ \  Flex,\ \ Bison,\ \ MIPS,\ \ REST \\
\textsc{Databases} & MySQL,\ \ SQLite\\
\textsc{Systems / Platforms} & Linux,\ \ Android,\ \ Windows,\ \ Git \\
% \textsc{Markup / Templating} & HTML, CSS \\
\textsc{Others} & Bash,\ \ Latex,\ \ Solidworks \\
\end{tabular}

%----------------------------------------------------------------------------------------
%	Experience
%----------------------------------------------------------------------------------------
\vspace{0.2cm}
\section{\textcolor{primary}{Research Experience}}
\begin{tabularx}{\linewidth}{ l | X }
\textsc{Feb 18} & \textbf{Artificial Intelligence Team Member} \hfill\href{http://www.agv.iitkgp.ac.in/}{\textbf{Autonomous Ground Vehicle Research Group}}\\
\textsc{May 19} & {- Worked as a software stack team member, tackling the various challenges faced to model a complete autonomous vehicle capable of traversing dynamic environments.}\\
& {- Worked on a novel and robust path planning algorithm taking the kinetic constraints of the bot into account for a dynamic environment supported by a high accuracy localization.}\\
& {- Responsible for the ideal integration of the various modules such as vision, localisation, planning, sensor data, etc.}
\end{tabularx}

%----------------------------------------------------------------------------------------
%	Projects
%----------------------------------------------------------------------------------------

\vspace{0.2cm}
\section{\textcolor{primary}{Projects}}
\vspace{-0.6cm}
\begin{tabular}{p{19.7cm}}
\begin{description}[style=nextline, font=$\bullet$\hspace{2mm}\normalsize]
%  \item[\href{https://github.com/TheLethalCode/opensoft18}{DigiCon},\space OpenSoft 2018 IIT Kharagpur]
%   - This web application accurately parses and mines the contents of a hand-written doctor's prescription with the help of OpenCV and advanced Optical Character Recognition(OCR).\newline - It can segregate the medicines along with their doses while checking for possible errors with the help of an advanced and optimized spellcheck with the support of medical texts and dictionaries.\newline
  
  \item[\textcolor{extra}{Hybrid A-Star},\space Path Planning Algorithm] - An improved, parallelized and novel implementation of the conventional Hybrid A-star global path planner using Dubins, Reeds-Shepp Path and Djikstra as heuristics while accounting for the kinetic constraints of the bot.\newline - It is capable of running at a frequency of 30Hz in a moderately populated environment and can achieve better results in a sparsely populated surrounding.\newline
  
  \item[\textcolor{extra}{Face Recognition},\space ANN Project]
  - This program identifies the direction the person in the image is facing with an accuracy of over 0.85.\newline - It uses an Artificial Neural Network (ANN) with three hidden units trained using back propagation algorithm over a training set of 500 images.\newline
  - This program can be reused for other similar classifications such as the person's expression, name, etc.\newline
 
  
 \item[\textcolor{extra}{Eklavya 6.0}, \space Intelligent Ground Vehicle Competition (IGVC) 2018]
 - A robot capable of intelligently traversing an obstacle ridden course while being restricted to a narrow lane in a restricted environment.\newline - It uses camera for the visual input, LIDAR for avoiding collisions, and other sensory inputs for controlling the position and velocity of the bot, which are implemented over the ROS Framework using various algorithms.\newline- The bot took part in the \href{http://www.igvc.org/}{\textbf{IGVC 2018}} competition for autonomous vehicles and bagged the 2nd place.\newline

\item[\textcolor{extra}{TinyC Compiler}, \space Compiler Design]
- A compiler designed using the Lexical Grammar and the Phase Structure Grammar provided for TinyC (a subset of C language) according to the \textbf{International Standard ISO/IEC 9899:1999 (E)}.\newline
- It uses Flex as a lexical analyzer and Bison for implementing the semantic actions and finally produces a Machine Independent Code and Three Address Code for a given source program. 
\end{description}
\end{tabular}

%----------------------------------------------------------------------------------------
%	INTERESTS   
%----------------------------------------------------------------------------------------
\vspace{-0.4cm}
\section{\textcolor{primary}{Technical Interests}}
\begin{description}
Algorithms and Data Structures, Number Theory, Machine Learning, Reinforcement Learning, Cryptography and Networking.
\end{description}


%----------------------------------------------------------------------------------------
%	COURSEWORK
%----------------------------------------------------------------------------------------

\vspace{0.1cm}
\section{\textcolor{primary}{Related Courses}} 
%	\hfill{\normalsize{* Currently Studying}}}
\vspace{-2.5mm}
\begin{itemize}
\begin{multicols}{2}
\setlength{\itemsep}{-2pt}
 \item Programming and Data Structures (T/L)
 \item Algorithms and Data Structures (T/L)
 \item Discrete Structures
 \item Software Engineering (T/L)
 \item Formal Language & Automata Theory
 \item Probability and Statistics
 \item Switching Circuits (T/L)
 \item Algorithms - II
 \item Computer Organization and Architecture (T/L)
 \item Compilers (T/L)
 \item Cryptography and Network Security
 \item Machine Learning
 \item Computer Networks (T/L) *
 \item Operating Systems (T/L) *
 \item Deep Learning *
 \item Principles of Programming Languages*
 \item Reinforcement Learning *
 
 \end{multicols}
\end{itemize}

%----------------------------------------------------------------------------------------
%	ACHIEVEMENTS AND INVOLVEMENTS
%----------------------------------------------------------------------------------------
\vspace{-0.1cm}
\section{\textcolor{primary}{Achievements \& Involvements}}
\vspace{-0.6cm}
\begin{tabular}{p{19.7cm}}
\begin{description}[font=$\bullet$\hspace{2mm}\normalsize]
 \item[\textcolor{extra}{Projects -}] I have taken up a lot of small scale projects like \newline
 - \textbf{DigiCon}, A web application that accurately parses and mines the contents of a hand-written doctor's prescription with \vspace{0.15cm} the help of OpenCV and advanced Optical Character Recognition(OCR) and offers it to the user in a readable fashion. \newline
 - \textbf{KGP-RISC Processor}, a processor with an ISA similar to that of MIPS, designed in Verilog and simulated in FPGA with a BRAM module. \vspace{0.15cm} It has a clock frequency of 1GHz and executes an average of 1 instruction every 4 clock cycles. \newline
 - \textbf{DWA Planner}, an improved DWA local path planning algorithm implemented with a new optimization function which reduces the chances of the algorithm failing in case of encountering a local minima and improves the traversal time. It \vspace{0.15cm}has a ROS interface.\newline
 - \textbf{Gymkhana Sports Management System}, a graphical application software that simplifies the registration and payment process for the various sports activities \vspace{0.15cm} hosted in Gymkhana.  \newline 
 - \textbf{Artemis Arrow}, a web application for scraping various forms of entertainment like novels, songs, anime, etc and offering the user a downloadable \vspace{0.15cm} link for the same. \newline 
 - \textbf{BrkOut}, a game developed using \vspace{0.15cm}Python incorporating real-time physics. \newline
 - \textbf{G\_Inface}, a command line interface for Google Drive developed using the REST API, useful for syncing personal local data with Google Drive. \newline
 
 \item[\textcolor{extra}{Competitive Coding -}] Participated in numerous contests and solved a wide range of problems hosted in the various Competitive Coding platforms such as \href{https://codeforces.com/profile/TheLethalCode}{CodeForces}, \href{https://www.codechef.com/users/thelethalcode}{CodeChef},
 \href{https://www.spoj.com/status/kosaksi}{SPOJ}, etc. I have also organized a lot of Competitive Programming contests (CodeNites) on Hackerearth in a university level.\newline
 
 \item[\href{https://kwoc.kossiitkgp.org/}{Kharagpur Winter of Code -}]One of the organizers of the five week long GSOC-styled Open Source Program and an active mentor of one of the projects in it. Responsible for the development and maintenance of the KWOC website.\newline
 
 \item[\textcolor{extra}{Kshitij Events, Technical Fest -}] Participated in various events like Source Code, Robotics, etc. and secured standings among them. The variety of events organized helps broaden one's horizon.\newline
 
 \item[\textcolor{extra}{Programming Societies -}] Co-founder of \href{https://www.facebook.com/codestashkgp/}{\textbf{CodeStash, IIT KGP}} and was a Core-Team Member of societies such as \href{https://kossiitkgp.in/}{\textbf{Kharagpur Open Source Societies }}and \href{https://www.facebook.com/CodeClub.IITKGP/}{\textbf{CodeClub}} which help to capture, nurture and preserve the programming zeal that bubbles among the budding KGP students by organizing workshops, hackathons, fests, etc.\newline
 
 \item[\textcolor{extra}{Scholastic Achievements}]\textbf{}
 \newline - \space \textbf{AIR 322 - JEE Advanced} (99.8 percentile) \newline - \space \textbf{AIR 1784 - JEE Mains} (99.8 percentile) \newline - \space Twice \href{http://www.kvpy.iisc.ernet.in/main/index.htm}{\textbf{Kishore Vaigyanic Protsahan Yojana (KVPY)}} Scholar
\end{description}
\end{tabular}
\end{document}

